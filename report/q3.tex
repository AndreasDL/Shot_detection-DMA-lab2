\section[A comparative discussion of the precision and recall values obtained for each video shot detection method, taking into account different parameter settings.]{}
The recall and precision of all different implement methods are depicted in Table \ref{tab:precisionrecall} in Appendix \ref{app:q3-1}. Overall, the Twin Comparison method seems to score the best on the Star Wars sequence. In this question, this method will be used as an example to show the difference when changing parameters.

Changing thresholds has an impact on the precision and recall values. By lowering the thresholds, more cuts will be detected, resulting in lower recall and higher precision. This is so because false posive detections will also be amongst the frames that weren't detected with a lower threshold. Table \ref{tab:precisionrecall} confirms this statement.
 
Graphs showing the results for the CSI sequence using the Twin Comparison method are added in Appendix \ref{app:entire} and \ref{app:zoom}. Figure \ref{fig:entire} shows the difference between subsequent frames for the entire video. The lower and higher threshold, ground truth and detected shots are also shown in the figure. In order to show the results more precise, Figure \ref{fig:zoom} zooms in on frames 0-800. One can notice that in these sequence all detected cuts are hard cuts, where the difference exceeds the higher threshold. Although the graph shows satisfiable results, there are still some faults in the detection.
 
The first mistake lies around frame 400, where the difference exceeds the lower threshold (but does not exceed the higher). These differences sum up to the higher threshold, resulting in a wrongly detected cut (= a false positive). The cause for this is a highly turbulent sequence of frames: a toilet being flushed. A print screen of this is shown in Figure \ref{fig:flush} in Appendix \ref{app:turbulent}.

The second noticeable mistake lies around frame 780. Here, two shots dissolve into each other, as shown in Figure \ref{fig:dissolve} in Appendix \ref{app:dissolve}. The dissolve stars around frame 750, which explains the increase in difference starting from that frame. The difference however does not exceed the lower threshold, which causes a false negative: the shot at frame 785 is not detected by the Twin Comparison. If the lower threshold would be lowered, the system will be able to detect this transition, but it will also include the detection of new false positives, probably somewhere around frames 90, 580 and 650.

Also, note that for some shots several high differences appear. An example of this is the hard cut at frame 111. Here, some other peak values appear after this frame. To filter these out, the system ignores all cuts within 10 frames of the previous cuts. The graph shows that this intervention filters out all these 'burst detections'.
\\