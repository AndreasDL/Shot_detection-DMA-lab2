\section[A summarizing ROC curve for each video sequence, illustrating the effectiveness of the different techniques implemented (see slide 26 of the introductory lecture). Corresponding tables need to be provided as well, containing the optimal parameter settings and the resulting precision, recall, and F1 score (the harmonic mean of precision and recall).]{}

The tables are included in \ref{ROCTable1} and \ref{ROCTable2} of Appendix \ref{app:q5}. The curves are in Figure \ref{rocPixel}, Figure \ref{rocMotion}, Figure \ref{rocGlobalLocal} and Figure \ref{rocTwin} of Appendix \ref{app:q5}. The optimal values, precision, recall and F1 values can be found in Table \ref{ROCValues} in Appendix \ref{app:q5}. 

It can be concluded that these methods all scale differently, but one common trend is observed. A higher recall usually means that the precision is lower. This is explained as follows, a higher recall corresponds with higher thresholds. This means that not all shots are detected, but the probability that the detection really is a shot, is higher. The recall / precision is a trade-off that should be chosen with care. On one hand, one can either go for a trustworthy method, that does not detect all shots, with the benefit of having less false detections. On the other hand a sensitive method, that finds more shots at the cost of having some false detections is also possible.
\\